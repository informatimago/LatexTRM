\documentclass[a4paper,oneside,12pt]{article}
\usepackage{longtable,geometry}
\usepackage[english]{babel}
\usepackage[utf8]{inputenc}
\usepackage[T1]{fontenc}
\usepackage[babel]{csquotes}
\usepackage{amsmath}
\usepackage{amssymb}
\usepackage{tikz}
\usepackage{pgfplots}
\usepackage{empheq}
\MakeAutoQuote{“}{”}
\pagestyle{headings}
\geometry{dvips,a4paper,margin=1.5in}
\setlength{\parskip}{0.2cm}
\title{Currency Relative Theory \\
\begin{displaymath}CRT \sum_{k=1}^{7}\frac{1}{k!}\end{displaymath}}

%% \author{cc-by-sa Stéphane Laborde}
%% translation: Pascal J. Bourguignon <pjb@informatimago.com>


\author{
  cc-by-sa Stéphane Laborde 
  \footnote{\small This translation of: 
``Théorie Relative de la Monnaie'' by Stéphane Laborde, 
\texttt{http://wiki.creationmonetaire.info/index.php?title=File:LaTexTRM.zip}
by: Pascal J. Bourguignon \texttt{<pjb@informatimago.com>} }
}

\begin{document}
\maketitle
\newpage
\tableofcontents
\newpage

\section{The Four Economic Freedoms}

For the Currency Relative Theory (CRT), the definition of freedom is
``what it is possible to achieve without harming yourself and
others''.  This doesn't include unconscious possibilities.

The CRT defines four economic freedoms which are at the root of its
approach.  They are:


\begin{enumerate}
\item The freedom of choice of one's own monetary system.
\item The freedom to use resources.
\item The freedom to estimate and produce any economic value.
\item The freedom to exchange, account and display prices using this ``currency''.
\end{enumerate}

Notably, the third freedom establishes the principle of relativity as
the essence of this approach. 


\section{Principle of Economic Relativity}

The CRT is based on the principle of economic relativity, which states
that each human being defines a legitimate referential to estimate and
produce all types of economic value, known or unknown of others. 

In other words there is no absolute economic value, no human being who
is legitimately able to define what is value or non-value for other
human beings be it in space (between present human beings) or in time
(between remote people over time). 


\section{Space-Time}

The economic space-time is characterized essentially by human beings
who belong to a given economic area.

The following thought experiment helps to understand this point: If we
remove from a given economic area some specific economic value, there
will always be an economic area.  On the other hand, if we remove all
the human beings from it, then there remains no observer, no actor in
this economic area.

This is therefore the human being who is the only invariant basis of
any economy.

But human beings are not absolute either, since they have a finite
average lifespan ``ev'' (average life expectancy), and they're
renewed in time, newborns replacing the dead.

This quantity is a finite data of the economic space-time studied by
the CRT where, for each time t, the set of human beings is renewed at
time t+ev.

In the following, we will call ``space'' the set of individuals for a
given date ``t'', and ``time'' the phenomenon of successive
replacement of those individuals in time.  Therefore, space-time must
be understood relatively to these definitions.

\section{Free Currency}

A currency is a reference economic value which gives a common metric,
for a given time and a given currency area, allowing measurement in the
same unit of values and exchanges, to facilitate the flow of the
economy between different actors.

Note that even though people do not agree on economic values either in
space or in time, they still use the same unit of individual valuation
in relation to a reference value, which is called ``currency''.

A currency area is defined by the set, depending on time E(t),
consisting of individuals I(x,t) who have adopted the same currency
(a given currency area may also deal with several currencies).

A currency is said to be ``free'' if it's a valid reference value for
a metric that respects the principle of relativity of all economic
value, and the human space-time defined above, not establishing any
arbitrary control on each other (which means that laws must be of the
same form for all), mainly with regard to the recognition and
production of any economic value.


To qualify as free, a currency, cannot be based on an arbitrary
decision as to what is value or non-value, or be created
preferentially for some human being in space, or in time.

It must be the accounting unit, because it is the reference of the
metric (as in relativist physics, speeds are expressed in proportion
of the speed of light).

Still, it must be an economic value (just like light is a physical
object), because we must have an economic metric.  But to be
independent of the other values, its production cost should be minimal
(the mass of light is zero, which is precisely what gives it its speed
invariance).

Therefore we must reconcile invariance and finiteness for the currency
as well as the minimal cost of production.  Since human beings are the
only invariant basis, it can only be a purely numerical quantity,
co-produced by human beings, a value expressed relatively to its own
total.

Let's call  $\left( \frac{M}{N} \right) (t)$ the average money M for
the N human beings of finite lifespan, stakeholders in this economy at
time ``t''.  The human beings must all be co-producer of this economic
value, while they are being replaced in time;  therefore we must
define a production of our reference value M, of same form for all
individuals, in space and in time.


Thus we establish an economic metric, whose reference value is
produced in an way that is invariant by referential changes
(change of individual, whatever the time when he's born, lives and
dies).

For each of those N individuals I(x,t) in the currency area thus
established, and on condition of quasi-stability (especially of N),
the relative instantaneous production (differential) of a free
currency, can only be identical in space (spacial symetry),  and
identical in time (time symetry).

Said otherwise, the production of a free currency can only be the same for
all the stakeholder individuals for a given time ``t'', and this
relative production is independent on time.

\begin{empheq}[box=\fbox]{equation}
\frac{d^{2} \left( \frac{M}{N} \right) }{dt \, dx}=0 \, \, \, \, and \, \, \, \, \frac{d \left( \frac{M}{N} \right) }{\left( \frac{M}{N} \right)}=c \, dt
\end{empheq}

In the following, we will drop the differential ``dt'', since dt = 1
with discrete transform.

We deduce, placing us under the assumption of continuity and
differentiability, (see the chapter "Variations of N and calculation
of UD"):

\begin{empheq}[box=\fbox]{equation}
\left( \frac{M}{N} \right) (t)=\left( \frac{M}{N} \right) (t_{0}) \, e^{ct}
\end{empheq}

On the other hand, the individuals having a finite life expectancy
``ev'', the instantaneous production (derivate) being established as
invariant, the relative individual sum produced during the life must
not be dependent on time either.

The currency of those who go must give way to the currency of those
who will replace them at the end of that period.  Which is equivalent
to say that  $\left( \frac{ev}{2} \right)$  years later, the
living must have co-produced their own full share of relative currency:

\begin{empheq}[box=\fbox]{equation}
\frac{\left( \frac{M}{N} \right)(t)}{\left( \frac{M}{N} \right)(t+\frac{ev}{2})}=e^{-c \left( \frac{ev}{2} \right)}
\end{empheq}

This symetry principle between those who go and those who arrive
establishes a convergent center of symetry at the point $\left(\frac{ev}{2} \right)$ 
where those who arrive at this point represent a proportion 
$\frac{1 \, year}{\left(\frac{ev}{2}\right)}$ of those who go; for an
other expression, see also (14):

\begin{empheq}[box=\fbox]{equation}
\frac{\left( \frac{M}{N} \right)(t)}{\left( \frac{M}{N} \right)(t+\frac{ev}{2})} = \frac{1 \, year}{\left( \frac{ev}{2} \right)}
\end{empheq}

From (1) and (4) we obtain a symetric rate where he average 
$\left(\frac{M}{N} \right)$ is reached for all individuals, 
close to $\frac{1 \, year}{\left(\frac{ev}{2}\right)}$, on the point
$\frac{1 \, year}{\left(\frac{ev}{2}\right)}$ of his participation in
the free currency thus established, whatever the time considered.

\begin{empheq}[box=\fbox]{equation}
c_{sym}=\frac{\ln(\frac{ev}{2})}{(\frac{ev}{2})}
\end{empheq}

The rates ``c'' that are inferior to $c_{sym}$ establish a metric that
favours  the older individuals, while the rates that are superior to
$c_{sym}$ favour the youger individuals.

This converging rate has a lower limit $c_{min}$ obtained for a
convergence reached at the end of the average life expectancy:

\begin{empheq}[box=\fbox]{equation}
c_{min}=\frac{\ln(ev)}{ev}
\end{empheq}

Numerical application for France with a life expectancy of 80 years in 2014:

\begin{empheq}[box=\fbox]{equation}
c_{sym}=\frac{\ln(40)}{40}=9,22 \% /an \,\,\,\, et \,\,\,\, c_{min}=\frac{\ln(80)}{80}=5,48 \% /an
\end{empheq}


\section{Quantitative}

We will call Universal Dividend the differential invariant quantity at
time ``t'' that we can describe either in continuous or discrete form
(which will be useful to establish approximations in practice):

\begin{displaymath}UD(t)=d \left( \frac{M}{N} \right) (t) = c \, \left( \frac{M}{N} \right) (t_{0}) \, e^{ct} \end{displaymath}

Or:

\begin{displaymath}
UD(t+dt) = UD(t) + dUD(t) = (1+c) UD(t)
\end{displaymath}

corresponding to the currency units co-created by the individuals for
the yearly unit of time ``t'', and which will be of the form:

\begin{empheq}[box=\fbox]{equation}
UD=c \left( \frac{M}{N} \right)
\end{empheq}

And Q(t) being the sum of the currency units co-produced by an
individual between the time $t_{0}$ start of his participation in the
free currency, and t:

\begin{empheq}[box=\fbox]{equation}
Q(t-t_{0})=\int_{t_{0}}^t UD(t) \, dt = \left( \frac{M}{N} \right)(t_{0}) \, e^{ct} \left(1 - e^{-c(t-t_{0})} \right)
\end{empheq}

Which gives, graphically:

\begin{tikzpicture}
\begin{semilogyaxis}[width=15cm,height=8cm,xmin=0,xmax=85,ymax=1000000000,axis x line=bottom,axis y line = left,xlabel={Year},ylabel={$Q(t-t_{0})=\int_{t_{0}}^t UD(t) \, dt$},legend entries={$Q(t-t_{0})=\int_{t_{0}}^t UD(t) \, dt$},legend style={at={(0.02,0.98)},anchor=north west}]
\addplot+[mark=none,fill=red,draw=red] file {Quantitatif.txt}\closedcycle;
\end{semilogyaxis}
\end{tikzpicture}

\newpage

\section{Relative}

Given the previous, we also have the following relative expression
for the reference currency of the global economic metric, under the
immutable form in space-time:

\begin{empheq}[box=\fbox]{equation}
\frac{M}{N}=\frac{1}{c} UD
\end{empheq}

And

\begin{displaymath}UD(t)=d \left( \frac{M}{N} \right) (t) = c \, \left( \frac{M}{N} \right) (t_{0}) \, e^{ct} \end{displaymath}


So we can also transform our metric based on the relative unit ``UD''
well established. Let's call now $R=\frac{Q}{UD}$ the number of
relative units co-produced by an individual between $t_{0}$ and t:

\begin{empheq}[box=\fbox]{equation}
R(t-t_{0})=\frac{\int_{t_{0}}^t UD(t) \, dt}{UD(t)}=\frac{1}{c}(1-e^{-c(t-t_{0})})
\end{empheq}

This gives, graphically:

\begin{tikzpicture}
\begin{axis}[width=15cm,height=7cm,xmin=0,xmax=85,axis x line=bottom,
axis y line = left,xlabel={Year},ylabel={$R(t-t_{0})=\frac{1}{c}(1-e^{-c(t-t_{0})})$},legend entries={$R(t-t_{0})=\frac{1}{c}(1-e^{-c(t-t_{0})})$},legend style={at={(0.6,0.15)},anchor=north west}]
\addplot+[mark=none,fill=red,draw=red] file {Relatif.txt}\closedcycle;
\end{axis}
\end{tikzpicture}

In the relative referential, the share of currency co-produced by any
individual participating in this metric converges asymptotically and
consistently (in space-time) toward:

\begin{empheq}[box=\fbox]{equation}
\lim\limits_{t \to {+\infty}} R(t-t_{0}) = \frac{1}{c}
\end{empheq}

And more specifically, for $t=t_{0}+\frac{ev}{2}$ with $c=\frac{\ln \left( \frac{ev}{2} \right) }{ \left( \frac{ev}{2} \right) }$ :

\begin{empheq}[box=\fbox]{equation}
R \left( \frac{ev}{2} \right)=\frac{1}{c} \left(1 - e^{-c\frac{ev}{2}} \right) = \frac{1}{c} \left(1 - \frac{1}{\left(\frac{ev}{2}\right)} \right)
\end{empheq}

Given (10), (11) and (13), we can express the fundamental expression
(4) as:

\begin{empheq}[box=\fbox]{equation}
\frac{\int_{t_{0}}^{t_{0}+\frac{ev}{2}} UD(t) \, dt}{\left( \frac{M}{N}\right) (t_{0}+\frac{ev}{2})}=\left(1 - \frac{1}{\left(\frac{ev}{2}\right)} \right)
\end{empheq}

which we may express, according to (14) as:

"The sum of the UD produced by an individual participating in a free
currency, during $\left(\frac{ev}{2}\right)$ converges toward the
average money supply at approximatively
 $\frac{1 \, year}{\left(\frac{ev}{2}\right)}$, whoever this individual
is and whatever the time considered."


Or also, according to (13) as:

"The sum of the relative UD produced by an individual particiapting of
a free currency, during $\left(\frac{ev}{2}\right)$ converges toward
$\frac{1}{c}$ at approximately
$\frac{1 \, year}{\left(\frac{ev}{2}\right)}$ 
whoever this individual is, and whatever the time considered."


Relative graph of the share of currency generated by an individual
during and after his leaving:

\begin{tikzpicture}
\begin{axis}[width=15cm,height=7cm,xmin=0,xmax=160,axis x line=bottom,
axis y line = left,xlabel={Year},ylabel={$R(t-t_{0})$},legend entries={$R(t-t_{0})$},legend style={at={(0.7,0.4)},anchor=north west}]
\addplot+[mark=none,fill=red,draw=red] file {Relatif_160ans.txt}\closedcycle;
\end{axis}
\end{tikzpicture}

\section{Initial Asymetries}

Let's consider the specific case of an individual starting his
participation in the metric with an initial share of currency (gift,
inheritance or any economic exchange) $Q_s(t_{0})$ and having exchanges
with the exterior that are balanced (monetary purchases are always
equal to the monetary sales).  This individual, called
pseudo-autonomous, will see his share of currency $Q_s(t)$ evolve as:

\underline{Quantitatively:}

\begin{displaymath}Q_s(t)=Q_s(t_{0})+\int_{t_{0}}^t UD(t) \, dt = Q_s(t_{0})+ \left( \frac{M}{N} \right)(t_{0}) \, e^{ct} \left(1 - e^{-c(t-t_{0})} \right) \end{displaymath}

\underline{Relatively} let's call $R_s(t)$ the evolution of his share
of currency:

\begin{displaymath}R_s(t)=\frac{Q_s(t_{0})+\int_{t_{0}}^t UD(t) \, dt}{UD(t)}=\frac{Q_s(t_{0})}{UD(t)}+\frac{1}{c}(1-e^{-c(t-t_{0})})\end{displaymath}

And we have:

\begin{displaymath}
UD(t)=UD(t_0) \, e^{c(t-t_{0})} \,\,\, and \, also \,\,\, R_s({t_0})=\frac{Q_s(t_{0})}{UD(t_{0})}
\end{displaymath}

Thus, factorizing we obtain finally the relative form:

\begin{empheq}[box=\fbox]{equation}
R_s(t)=\frac{1}{c}\left[ 1-e^{-c(t-t_{0})}\left( 1-cR_s(t_{0}) \right) \right]
\end{empheq}

Where we see directly that if $R_s(t_{0})=\frac{1}{c}$ which is
equivalent to ${Q_s(t_{0})}=\left( \frac{M}{N} \right) (t_{0})$, then
for all t we will have: \begin{displaymath}R_s(t)=\frac{1}{c}\end{displaymath}

Now, depending on the three cases, $R_s(t=t_{0})<\frac{1}{c}$, 
$R_s(t=t_{0})=\frac{1}{c}$ or $R_s(t=t_{0})>\frac{1}{c}$, 
we have, under condition of balanced exchanges, the following three
evolutions in the relative referential:

\begin{tikzpicture}
\begin{axis}[width=15cm,height=7cm,xmin=0,xmax=85,axis x line=bottom,
axis y line = left,xlabel={Year},ylabel={$R_s(t-t_{0})$},legend entries={$R1_s(t=t_{0})<\frac{1}{c}$,$R2_s(t=t_{0})=\frac{1}{c}$,$R3_s(t=t_{0})>\frac{1}{c}$},legend style={at={(0.7,0.9)},anchor=north west}]
\addplot table[x index=0, y index=1]{3Asymetriques.txt};
\addplot table[x index=0, y index=2]{3Asymetriques.txt};
\addplot table[x index=0, y index=3]{3Asymetriques.txt};
\end{axis}
\end{tikzpicture}

This evolution is only valid in this specific case studied here.


\section{The four referentials}

We've seen previously two metric referentials, quantitative and
relative, whose transformation law is given by:

\begin{displaymath}R_s(t-t_{0})=\frac{Q_s(t-t_{0})}{UD(t)} \end{displaymath}

We may also establish the quantitative referential with zero-sum of
accounts, with the transformation:

\begin{displaymath}Z_q(t-t_{0})=Q_s(t-t_{0})-\left( \frac{M}{N} \right) (t) \end{displaymath}

and also the relative referential with zero-sum of accounts:

\begin{displaymath}Z_r(t-t_{0})=\frac{Z_q(t-t_{0})}{UD(t)}=R_s(t-t_{0}) - \frac{1}{c} \end{displaymath}

All individuals being perfectly able to switch to the referential that
seems best adapted.  The same free currency system can propose at
least four distinct referentials for each individual participating in
it, this choice being purely personal:

\begin{enumerate}
\item The quantitative referential.
\item The quantitative referential with zero-sum.
\item The relative referential.
\item The relatiev referential with zero-sum.
\end{enumerate}

\section{Variation for a pseudo-autonomous individual}

Let's study here the variation of a monetary account for a
pseudo-autonomous individual.  First, in quantitative:

\begin{displaymath}dQ_s(t)=UD(t) \end{displaymath}

And in relative:

\begin{displaymath}dR_s(t)=e^{-c(t-t_{0})} \left( 1-cR_s(t_{0}) \right) = 1-cR_s(t) \end{displaymath}

This let us affirm the following conclusions (a) and (b), perfectly equivalent:

\paragraph*{(a)}

``In the quantitative referential, the account of a pseudo-autonomous
individual appears as if a Universal Dividend is added between two
time units.''

\paragraph*{(b)}

``In the relative referential, the account of a pseudo-autonomous
individual appears as if between two units of time, one Universal
Dividend was added, and at the same time, a proportion equal to c was
removed.'' 


Understanding that these points are only appearance, an individual
participating in a free currency chooses the referential of his choice
in terms of his monetary accounts, quantitative, relative,
quantitative zero-sum, relative zero-sum, or another referencial he
deems most consistent with his experience, this in no way affecting
the free currency established.

\section{Variations of N and calculation of UD}

Given the above it should be borne in mind that it is the convergence
at half-life which is the goal sought by a free currency, new
entrants replacing the dead (see in this connection forms (4) and (14)
concerning the time condition valid for any individual). 

When looking for a practical calculation method of UD, we must not
make an estimation using only the local differential calculus.  We
must keep in mind the fundamental behavior of a free currency, which
is to ensure for each human being, during his life, and particularly
at the center of the temporal symetry, at half-life, the same relative
share of currency as his predecessors and successors at the same point.

In particular we will be convinced by the reflection of the need to
address the practical solution by considering extreme cases, such as
the case of a sharp rise in the number of participants in a free
currency (equivalent to a pseudo-initialization of money), where UD
calculated in relative ($UD(t)=c\left(\frac{M}{N}\right)(t)$) suffer a
strong discontinuity, destroying the continuity of the progression,
and would become extremely low with respect to the initial
participants, few, and who possess in this case a monetary share
extremely strong compared to new entrants, unrelated to the computed
UD.

Said otherwise, in a more mathematical way, the fundamental equations
(1) and (4) expressed in the analysis of the form of a free currency,
have identified solutions only for a continuous and derivable 
$\left(\frac{M}{N}\right)$ (or quasi-continuous and quasi-derivable),
which we'll have to approximate to the best in case of discontinuous
variations.

In this reflection we need to have initially a UD(t=0) not relative,
since to establish a monetary amount, it is still necessary that the
currency exists first.  We understand that in this case there is a
convergence of phenomenon between the initialization of a free
currency and the huge increase in the number of members of an
installed currency.  The solution complying with the CRT having to be
independent of time (principle of relativity), we now understand that
we must in these cases establish a non-relative amount of UD(t),
that is, a fixed and stable amount, until the relative area is reached.

N(t) is unknown, so to evaluate the shape of a general and practical
method of generation, we must establish a most simple and readable
method, which we can approximate thru the modelization of the
variation of N as $dN(t)=\alpha N(t)$ or also
$N(t+dt)=N(t)+dN(t)=(1+\alpha)N(t)$ and we take an approximation for M
conform to $M(t+dt) \approx (1+c) M(t)$.
\par

It should be noted that $\alpha$  must be understood as being in
general ``small'' on durations on the order of
 $\left( \frac{ev}{2} \right)$, and even compared to c.  Indeed, to
take the example of France between 1950 and 1999, the population went
from 41 to 56 millions, which corresponds to
$ \alpha = \frac{ln(\frac{56}{41})}{40} = 0,78 \%$/year while
$c=\frac{ln(40)}{40}=9,22 \%$/year.

We obtain an approximation of the differential variation of the Dividend:

\begin{displaymath} UD(t+dt)= c \, \frac{M(t+dt)}{N(t+dt)} \approx c \, \frac{(1+c)M(t)}{(1+\alpha)N(t)}\end{displaymath}

From which we deduct this first form:

\begin{displaymath}UD(t+dt) \approx \frac{(1+c)}{(1+\alpha)} UD(t)\end{displaymath}

As well as a second form approximated to the first order ("c" being small):

\begin{displaymath}UD(t+dt) \approx \frac{(c+c^2)M(t)}{N(t+dt)}\approx c \, \frac{M(t)}{N(t+dt)}\end{displaymath}

A simple lower bound appears for $\alpha$ positive, if 
$\alpha \approx c$ we have $UD(t+dt) \approx UD(t)$,
and another simple minimal bound appears for $\alpha$ 
small and negative, that we are happy to find under this form, since
it is very close to the definition: $UD(t) = c \, \frac{M(t)}{N(t)}$.

From these two minimum bounds revealed by this approximation we can
derive a simple practical calculation of UD, showing a quantitative
form and another relative, adapting flexibly to changes in N:

\begin{empheq}[box=\fbox]{equation}
UD(t+dt) = Max \left[ UD(t);c\,\frac{M(t)}{N(t+dt)} \right] \end{empheq}

In particular, we recognize that for a stable N, the form will
converge quickly toward its relative fundamental expression (which is
absolutely necessary):

\begin{displaymath} UD = c \, \frac{M}{N}\end{displaymath}

This form is notably extremely practical for the development of an
independent free currency from scratch, but also equivalently, to
manage in a flexible way the imprevisible variations of N, while
having an invariant law in space and time, and without moving away
from the basic form.

While being simple, easy to understand, and reassuring from a
quantitative point of view, this form appears as the best we may find.

We may summary the behavior thus:

``The UD never drops in quantitative, and is always at least equal to
a relative proportion c of the money supply.''

Other forms are of course possible, given the uncertainty of N(t), the
simpliest forms being better…

In general, to ensure the relevance of this form, and possibly to
compare it with others, such as the trivial but dangerous theorical
form  $UD(t+dt)=(1+c)UD(t)$,   it is necessary to simulate random N(t)
and to test the different forms, keeping in mind that to do that, we
must use individuals with limited lifespan, while simulating
operations for larger periods than ev, and evaluating if for all of
these individuals, the fundamental principles are respected, almost
all the time.

\end{document}
