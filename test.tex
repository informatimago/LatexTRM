\documentclass[a4paper,oneside,12pt]{article}
\usepackage{longtable,geometry}
\usepackage[frenchb]{babel}
\usepackage[utf8]{inputenc}
\usepackage[T1]{fontenc}
\usepackage[babel]{csquotes}
\usepackage{amsmath}
\usepackage{amssymb}
\usepackage{tikz}
\usepackage{pgfplots}
\MakeAutoQuote{«}{»}
\pagestyle{headings}
\geometry{dvips,a4paper,margin=1.5in}
\title{Théorie Relative de la Monnaie \\
\begin{displaymath}TRM \sum_{k=1}^{7}\frac{1}{n!}\end{displaymath}}
\author{Stéphane Laborde}

\begin{document}
\maketitle
\newpage
\tableofcontents
\newpage
\section{Principe de relativité économique}
La TRM se fonde sur le principe de relativité économique, qui établit que tout être humain est légitime pour estimer et produire tout type de valeur économique.

Autrement dit il n'y a pas de valeur économique absolue, pas d'être humain qui soit légitimement en mesure de définir ce qui est valeur ou non-valeur pour les autres êtres humains.
\section{Espace-Temps}
L'espace-temps économique est caractérisé essentiellement par les hommes qui font partie d'une zone économique donnée.

L'expérience de pensée suivante permet de comprendre ce point : Si on enlève d'une zone économique donnée une valeur économique particulière, il restera toujours une zone économique. Si par contre on y enlève les hommes, alors il ne reste ni observateur ni acteur de cette zone économique.

C'est donc l'homme qui est le seul fondement invariant de toute économie.

Mais les hommes ne sont par ailleurs pas absolus non-plus, puisqu'ils sont de durée de vie moyenne limitée notée "ev", et se renouvellent dans le temps, les nouveaux nés remplaçant les morts.

Cette dimension est une donnée finie de l'espace-temps économique étudié par la TRM où, pour tout temps t considéré, l'ensemble des hommes est renouvelé à la date t+ev.

\section{Monnaie Libre}

Une monnaie est une valeur économique de référence qui sert de métrique commune, pendant un temps donné et une zone monétaire donnée, pour ce qui concerne la mesure et la comptabilité des coûts, de la comptabilité et des échanges économiques.

Une zone monétaire étant définie par l'ensemble dépendant du temps E(t) composé des individus I(x,t) qui ont adopté cette même métrique (une même zone économique peut comporter plusieurs monnaies).

Une monnaie est dite "libre" s'il s'agit d'une métrique qui respecte le principe de relativité de toute valeur économique, ainsi que l'espace-temps humain défini ci-dessus.

Pour être qualifiée de libre une monnaie ne peut être qu'une valeur économique de coût minimal, voire nul, qui est adopté comme moyen universel d'échange et de comptabilité.

Elle doit être l'outil comptable parce qu'elle est la métrique.

Elle doit être une valeur économique tout de même, parce que nous devons avoir une métrique économique, mais devant être indépendante des autre valeur économique, son coût de production doit être minimal.

Il faut donc concilier finitude pour la valeur, et coût de production minimal pour la métrique, l'homme étant seul fondement invariant, il ne peut s'agir que d'une valeur purement numérique, co-produite par les hommes.

Nous appelons $\frac{M}{N}$ la production moyenne de cette valeur économique totale M pour les N hommes à durée de vie limitée partie prenante de cette économie.

Les hommes devant être tous co-producteurs de cette même valeur économique, alors qu'ils se remplacent dans le temps, nous devons donc définir une production de notre valeur de référence M constante, indépendante des individus et du temps.

Nous produisons ainsi une métrique économique dont la production est invariante par changement de référentiel (changement d'individu, quelle que soit l'époque à laquelle il naît, vit et meurt).

\begin{center}
\begin{displaymath}\frac{d \left( \frac{M}{N} \right) }{\left( \frac{M}{N} \right)}=c \end{displaymath}
\end{center}

D'où l'on déduit :

\begin{center}
\begin{displaymath}\left( \frac{M}{N} \right) (t)=\left( \frac{M}{N} \right) (t_{0}) \, e^{ct} \end{displaymath}
\end{center}

Mais par ailleurs les individus ayant une durée de vie limitée "ev", la création (dérivée) de notre métrique économique étant établie comme invariante, la somme relative individuelle produite pendant une durée de vie ne doit pas être non-plus dépendante du temps.

La monnaie de ceux qui s'en vont ne doit plus peser pour $\frac{ev}{2}$ que $\frac{1 an}{(\frac{ev2}{2})}$, ce qui est équivalent à dire que pour $\frac{ev}{2}$ les vivants ont co-créé leur part relative équivalente de monnaie à $\frac{1 an}{(\frac{ev2}{2})}$ près.

\begin{center}
\begin{displaymath} \frac{\left( \frac{M}{N} \right)(t)}{\left( \frac{M}{N} \right)(t+\frac{ev}{2})}=e^{-c \left( \frac{ev}{2} \right)} \end{displaymath}
\end{center}

Autrement dit La monnaie créée par les morts doit laisser place à la monnaie des vivants, et les vivants prendre possession de leur monnaie. Ce principe symétrique entre ceux qui s'en vont et ceux qui arrivent établit un centre de symétrie de convergence à $\frac{ev}{2}$.

Le centre de symétrie temporelle sera donc établi pour :

\begin{center}
\begin{displaymath}e^{-c \left(\frac{ev}{2}\right)} = \frac{1}{\left( \frac{ev}{2} \right)}\end{displaymath}
\end{center}

D'où il s'ensuit que nous obtenons un taux symétrique où la moyenne $\frac{M}{N}$ est atteinte lors de la demie vie des individus participant de cette même métrique :

\begin{center}
\begin{displaymath}c_{sym}=\frac{\ln(\frac{ev}{2})}{(\frac{ev}{2})}\end{displaymath}
\end{center}

Les taux "c" inférieurs à $c_{sym}$ établiront une métrique favorisant les individus plus âgés, tandis que les taux supérieurs favoriseront les individus les plus jeunes.

Ce taux de convergence a une limite basse $c_{min}$ obtenue pour une convergence atteinte en fin d'espérance de vie moyenne :

\begin{center}
\begin{displaymath}c_{min}=\frac{\ln(ev)}{ev}\end{displaymath}
\end{center}

\section{Quantitatif}
Nous appelons Dividende Universel la quantité différentielle à la date "t", que nous pouvons décrire indifféremment sous forme discrète ou continue :

\begin{center}
\begin{displaymath}DU(t)=\frac{M}{N}(t+1)-\frac{M}{N}(t)=d \left( \frac{M}{N} \right) (t)\end{displaymath}
\end{center}

Correspondant aux unités monétaires co-créée par les individus pour l'unité de temps annuelle "t", et qui sera donc de la forme :

\begin{center}
\begin{displaymath}DU=c \left( \frac{M}{N} \right) \end{displaymath}
\end{center}

Et Q(t) la somme des unités monétaires co-produite par un individu entre les instants $t_{0}$ date initiale de sa participation à la métrique et t :

\begin{center}
\begin{displaymath}Q(t-t_{0})=\int_{t_{0}}^t DU(t) \, dt = \left( \frac{M}{N} \right)(t_{0}) \, e^{ct} \left(1 - e^{-c(t-t_{0})} \right) \end{displaymath}
\end{center}



\section{Relatif}
Etant donné ce qui précède nous avons aussi l'expression relative de notre métrique économique globale sous la forme :

\begin{center}
\begin{displaymath}\frac{M}{N}=\frac{1}{c} DU\end{displaymath}
\end{center}

Et

\begin{center}
\begin{displaymath}DU(t)=d\frac{M}{N}(t) = c \frac{M}{N}(t_{0})\exp(ct)\end{displaymath}
\end{center}

Nous pouvons donc aussi transformer notre métrique en relatif sur la base de l'unité relative "DU" ainsi établie.

Appelons $R=\frac{Q}{DU}$ le nombre d'unités relatives co-créée par un individu entre $t_{0}$ et t :

\begin{center}
\begin{displaymath}R(t-t_{0})=\frac{\int_{t_{0}}^t DU(t) \, dt}{DU(t)}=\frac{1}{c}(1-e^{c(t_{0}-t)})\end{displaymath}
\end{center}

Autrement dit, dans le référentiel relatif la part de monnaie co-produite par tout individu participant de cette métrique converge asymptotiquement et invariablement (dans l'espace-temps) vers :

\begin{displaymath}\lim\limits_{(t-t_{0}) \to \frac{ev}{2}} R(t-t_{0}) = \frac{M}{N} \left(1-\frac{1}{(\frac{ev}{2})}\right) = \frac{1}{c} \left(1-\frac{1}{(\frac{ev}{2})}\right) DU \end{displaymath}.

\begin{tikzpicture}
\begin{semilogyaxis}[log basis y=2,grid=major,samples at={-4,...,4}]
\addplot {2^x};
\end{semilogyaxis}
\end{tikzpicture}
~
\begin{tikzpicture}
\begin{semilogyaxis}[log basis y=10,samples at={-4,...,4}]
\addplot {2^x};
\end{semilogyaxis}
\end{tikzpicture}


\end{document}